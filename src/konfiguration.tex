%!TEX root = ../hausarbeit.tex

\documentclass[12pt,a4paper]{article}
\usepackage[utf8]{inputenc}
\usepackage[ngerman]{babel}
\usepackage[T1]{fontenc}
\usepackage{amsmath}
\usepackage{amsfonts}
\usepackage{amssymb}
\usepackage{xspace}
\usepackage{graphicx}
\usepackage{color}

%%%%%%%%%%%%%%%%%%%%%%%%%%%%%%%%%%%%%%%%%%%%%%%%%%%%%%%%%%%%%%%%%%%%%%%%%%%%%%%%%%%%%%%%%%
% In diesem Abschnitt gibst du deine persönlichen Daten sowie den Kontext deiner Arbeit an
% das \xspace am Ende der Einträge muss vorhanden bleiben. Es regelt das Whitespace-
% handling
%%%%%%%%%%%%%%%%%%%%%%%%%%%%%%%%%%%%%%%%%%%%%%%%%%%%%%%%%%%%%%%%%%%%%%%%%%%%%%%%%%%%%%%%%%
% persönliche Daten
\newcommand{\vorname}{Hier dein Vorname\xspace}
\newcommand{\nachname}{Hier dein Nachname\xspace}
\newcommand{\emailadresse}{Hier deine Mailadresse\xspace}
\newcommand{\matrikelnummer}{Hier deine Matrikelnummer\xspace}


% Arbeitsspezifische Daten
\newcommand{\titel}{Hier der Titel der Arbeit\xspace}
\newcommand{\untertitel}{Hier der Untertitel der Arbeit\xspace}
\newcommand{\modulname}{Hier der Modulname\xspace}
\newcommand{\pruefer}{Hier der vollständige Name des Prüfers\xspace}
\newcommand{\semester}{Das Semester\xspace}
\newcommand{\ortderarbeit}{Hier der Ort für die Erklärung der Selsbtständigkeit\xspace}
%%%%%%%%%%%%%%%%%%%%%%%%%%%%%%%%%%%%%%%%%%%%%%%%%%%%%%%%%%%%%%%%%%%%%%%%%%%%%%%%%%%%%%%%%%

%Hurenkinder uns Schusterjungen
\clubpenalty10000
\widowpenalty10000
\displaywidowpenalty=10000

%Gimmmick: Linked Kapitel und Inhaltsverzeichnis, sowie Referenzen
\usepackage{hyperref}
\hypersetup{
    colorlinks,
    citecolor=black,
    filecolor=black,
    linkcolor=black,
    urlcolor=black
}

\usepackage{prettyref}
\usepackage{titleref}

%Schriftart
\usepackage{lmodern}
\usepackage{mathptmx}
\usepackage[scaled=.90]{helvet}

\usepackage{courier}
%Inhaltsverzeichnis
\usepackage[tocgraduated]{tocstyle}
\usetocstyle{allwithdot}
\setcounter{tocdepth}{3}

% Formatierungshilfen: http://wwws.htwk-leipzig.de/~myagovki/latex/formatierungshilfen/

\usepackage[ddmmyyyy]{datetime}
\renewcommand{\dateseparator}{.}

% Für Kopf und Fußzeile: https://esc-now.de/_/latex-individuelle-kopf--und-fusszeilen/?lang=en
\usepackage[
  headsepline, plainheadsepline,
  footsepline, plainfootsepline
]{scrlayer-scrpage}
\pagestyle{scrheadings}
\clearscrheadfoot

\ihead*{\titel}
\ohead*{\vorname\:\nachname}
\ifoot*{\modulname}
\ofoot*{\pagemark}
\AtEndDocument{\ofoot{}}

% anderthalbfacher Zeilenabstand
\usepackage[onehalfspacing]{setspace}

%Geometry der Seite festlegen
\usepackage{geometry}
\geometry{
  left=2.5cm,
  right=5cm,
  top=2.5cm,
  bottom=2cm,
  headheight=33pt
}

%Für Gestaltung des Layouts
\usepackage{blindtext}


%Bilbliothek und Literatur einbinden

\usepackage[style=authoryear-ibid,natbib=true,backend=biber,sorting=nty,hyperref=true]{biblatex}
\usepackage[babel,german=guillemets]{csquotes}
\addbibresource{bib/literatur.bib}


\newcommand{\zb}{z.\,B.\xspace}




